\documentclass[portrait, a0paper, margin=.5cm]{baposter}

\usepackage{poster_style}

\begin{document}
    \begin{poster}%
    {
        grid=false,
        eyecatcher=true,
        bgColorOne=white,
        bgColorTwo=white,
        borderColor=IGNGrey,
        headerColorOne=IGNGrey,
        headerColorTwo=IGNGrey,
        headerFontColor=white,
        boxColorOne=white,
        boxColorTwo=white,
        colspacing=.5em,
        columns=6,
        textborder=rounded,
        headerborder=closed,
        headerheight=0.12\textheight,
        headershape=rounded,
        textfont={\color{IGNDarkGrey}},
        boxshade=plain,
        background=none,
        linewidth=1pt
    }
    {}
    {
        \color{IGNDarkGrey}
        3D Building model self diagnostic\\ for automatic urban scene reconstruction evaluation
    }
    {
        \vspace{.5cm}
        \color{IGNDarkGrey}
        \begin{tabular}{c}
            \textsc{Oussama Ennafii}\\
            \small oussama.ennafii@ign.fr
        \end{tabular}
    }
    {
        \begin{tabular}{c}
            \includegraphics[width=2.2cm]{theme/ign_logo}\\~\\
            \includegraphics[width=2.2cm]{theme/paris_est_logo}
        \end{tabular}
    }

        \TransitionBox{motivation}{}{{\large \sc Motivation:} Why do we need automatic scene evaluation?}

        \StandardBox{auto_evaluation}{column=2, span=1, row=.03}{Self Diagnostic}{
            Can be used for:
            \begin{itemize}[label=--, leftmargin=1.5em]
                \item Change detection;
                \item Urban model correction;
                \item Urban reconstruction method evaluation;
                \item Crowd reconstruction quality assessment.
            \end{itemize}
        }

        \StandardBox{context}{column=0, span=2, aligned=auto_evaluation, bottomaligned=auto_evaluation}{Context}{
            \begin{itemize}[label=, leftmargin=*]
                \item Urban models have a wide range application range~\cite{Biljecki2015};
                \item Automatic urban modeling is an active research area~\cite{Musialski2012}, but {\color{IGNGreen}not yet operational};
                \item Example:
                \begin{itemize}[label=$\rightarrow$]
                    \item The IGN solution Bati3D takes more ressources to correct manually than an interactive reconstruction.
                \end{itemize}
            \end{itemize}
        }

        \StandardBox{situation}{column=3, span=3, aligned=auto_evaluation}{Overall perspective}{
            \includestandalone[mode=buildnew, width=\textwidth]{situation}
            \captionof{figure}{3D urban modelisation pipeline.}
        }

        \StandardBox{applications}{column=0, span=3, below=context}{Urban models applications}{
            \begin{center}
                \begin{tabular}{l l l}
                    \toprule
                        Planning & Simulation & Visualization \\
                        \midrule
                        City planning & Micro climates & Architecture \\
                        Emergency intervention & Wave propagation & Cadastre \\
                        Home decoration & Run-off water & Tourism \\
                        Communication network & Military intervention & Video games \\
                        \bottomrule
                    \end{tabular}
                \captionof{table}{\label{tab::3d_applications} Some of the main thematic applications of $3D$ urban reconstruction\cite{Biljecki2015, Scholze2002}.}
            \end{center}
        }
        \StandardBox{state_art}{column=3, span=3, below=situation, bottomaligned=applications}{State of the art}{
            Urban models quality assessment methods can be divided as follows:
            \begin{itemize}[label=--, leftmargin=1em]
                \item Methods that gives {\color{IGNGreen} geometric indexes} (for instance, heigh accuracy, completness \dots) by comparing to a {\color{IGNGreen} highier precision model}~\cite{Kaartinen2005, Zeng2014, Voegtle2003, Henricsson1997}.
                \item Methods that outputs end user oriented {\color{IGNGreen} topological and geometric error} using {\color{IGNGreen}remote sensing data} (LiDAR, DSM, Orthoimage)~\cite{Akca2010, OudeElberink2010, Boudet2006, Michelin2013}.
            \end{itemize}
        }

        \TransitionBox{formul}{below=applications}{{\large \sc Problem formulation:} We want to build the least reference dependent possible automatic method for urban model diagnostic.}

        \StandardBox{taxonomy}{column=0, span=4, row=.372}{Error taxonomy}{
            \includestandalone[mode=buildnew, width=\textwidth]{mind_map}
        }

        \StandardBox{errors}{column=4, span=2, aligned=taxonomy}{Error description}{
            \begin{enumerate}[label=(\roman*).]
                \item Unqualified Building Errors; not be taken into consideration:
                \begin{itemize}[label=-, leftmargin=.5em]
                    \item Half Building: partially reconstructed,
                    \item Changed Building: the building has changed,
                    \item Occlusion: occluded by vegetation,
                    \item Unknown: unverifiable shape;
                \end{itemize}
                \item Building Errors; building wise ($LoD0 \cup LoD1$) errors:
                \begin{itemize}[label=-, leftmargin=.5em]
                    \item Under Segmentation two or more buildings grouped into one,
                    \item Over segmentation: one building segmented into two or more buildings,
                    \item Footprint: wrong footprint,
                    \item Height: wrong building height;
                \end{itemize}
                \item Facet Errors; errors affecting only a facet:
                \begin{itemize}[label=-, leftmargin=.5em]
                    \item Under Segmentation: two facets or more grouped into one,
                    \item Over Segmentation: one facet segemented into two or more facets,
                    \item Imprecise Segmentation: imprecise facet edges,
                    \item Slope: imprecise slope.
                \end{itemize}
            \end{enumerate}
        }

        \StandardBox{info}{column=4, span=2, below=errors}{Informations}{
        \begin{itemize}[label=--, leftmargin=*]
            \item IGN funded;
            \item Cl\'ement Mallet \& Florent Lafarge codirection;
            \item Arnaud Le Bris supervision.
        \end{itemize}
        }

        \StandardBox{results}{column=0, span=4, below=taxonomy, bottomaligned=info}{Results}{
            \begin{multicols}{3}
                \captionof{table}{\label{tab::binary_rf_1000_4}Binary classification results using a Random Forests (trees: $1000$, depth: $4$).}
                \begin{tabular}{c c c}
                    \toprule
                    \multicolumn{3}{c}{\textbf{Cross validation results}}\\
                    \midrule
                    Metric & Max & Min \\
                     \midrule
                    train scores & $1.000$ & $0.9911$ \\
                     \midrule
                    test scores & $1.000$ & $0.8775$\\
                     \bottomrule
                \end{tabular}

                \captionof{table}{\label{tab::binary_rf_1000_4}Binary classification results using a Random Forests (trees: $1000$, depth: $4$).}
                \begin{tabular}{c c c}
                    \toprule
                    \multicolumn{3}{c}{\textbf{Cross validation results}}\\
                    \midrule
                    Metric & Max & Min \\
                     \midrule
                    train scores & $1.000$ & $0.9911$ \\
                     \midrule
                    test scores & $1.000$ & $0.8775$\\
                     \bottomrule
                \end{tabular}

                \captionof{table}{\label{tab::binary_rf_1000_4}Binary classification results using a Random Forests (trees: $1000$, depth: $4$).}
                \begin{tabular}{c c c}
                    \toprule
                    \multicolumn{3}{c}{\textbf{Cross validation results}}\\
                    \midrule
                    Metric & Max & Min \\
                     \midrule
                    train scores & $1.000$ & $0.9911$ \\
                     \midrule
                    test scores & $1.000$ & $0.8775$\\
                     \bottomrule
                \end{tabular}
            \end{multicols}
        }

        \ReferencesBox{references}{column=0, span=5, below=info}{\bf{References}}{
            \setlength{\columnseprule}{0.1pt}
            \begin{multicols}{3}
                \renewcommand{\section}[2]{}
                \bibliographystyle{abbrv}
                \tiny \bibliography{references}
            \end{multicols}
        }

        \InvisibleBox{qr}{column=5, span=1, aligned=references}{}{
            \begin{center}
                \includegraphics[width=\textwidth]{qr}
            \end{center}
        }

        \InvisibleBox{event}{column=2, span=2, row=.985}{}{
            \begin{tikzpicture}
                \node[rectangle, minimum width=\textwidth]{
                    \footnotesize{Journées de la Recherche 2018}
                };
            \end{tikzpicture}
        }
    \end{poster}
\end{document}
